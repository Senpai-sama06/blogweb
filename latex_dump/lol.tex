\documentclass{scrartcl}
\usepackage[sexy]{evan}
\usepackage[utf8]{inputenc}
\usepackage{graphicx}
\usepackage{asymptote}

\title{Library of Lemmas}
\author{Ramakrishna Sen}
\date{January 2021}

\begin{document}

\maketitle
\section{Power of a point lemma}
\begin{definition}
We define the power of a point $X$ which lies on the line segment $AB$ to be $AX\cdot XB$. This is equivalent to $OX^2-r^2$, where $O$ is the center of the circle.
\end{definition}
\begin{theorem}[Raw elementary]
The power of a point theorem just says that if two line segments $AB$ and $CD$ intersect at point $X$, and if $AX\cdot XB=CX\cdot XD$, then $ABCD$ is concyclic.
\end{theorem}
\section{Haruki's Lemma}
\section{Orders modulo a prime}
\begin{definition}[Orders modulo a prime]
\alert{The order of a number modulo prime} is the least positive integer $n$ such that $a^n \equiv 1\pmod{p}$
\end{definition}
\begin{lemma}[Result of negative residues.]
Let $n^2\equiv -1\pmod{p}$ for some odd primes, then $p\equiv -1\pmod{4}$
\end{lemma}
\begin{proof}
$n^2\equiv -1 \pmod{p}\Longrightarrow n^4\equiv 1\pmod{p}$ which means that 4 is an order modulo $p$. According to the fundamental principle of Orders, $4|p-1$.
\end{proof}
\newpage
\section{The butterfly of the garden}
\begin{lemma}[The butterfly lemma]
For a cyclic quadrilateral $ABCD$ with circumcircle $\omega$ let the diagonals of the quadrilateral $AC$ and $BD$ meet at $M$. Let there be another chord $EF$ such that the midpoint of the chord is $M$. Then the midpoint of $XY$ is also $M$.
\begin{center}
    \begin{asy}
         /* Geogebra to Asymptote conversion, documentation at artofproblemsolving.com/Wiki go to User:Azjps/geogebra */
import graph; size(8cm); 
real labelscalefactor = 0.5; /* changes label-to-point distance */
pen dps = linewidth(0.7) + fontsize(10); defaultpen(dps); /* default pen style */ 
pen dotstyle = black; /* point style */ 
real xmin = -13.66, xmax = 13.66, ymin = -6.52, ymax = 6.52;  /* image dimensions */
pen zzttqq = rgb(0.6,0.2,0); 

draw((-2.9,3.62)--(-4.76,-1.4)--(3.96,-2.58)--cycle, linewidth(1) + zzttqq); 
 /* draw figures */
draw((-2.9,3.62)--(-4.76,-1.4), linewidth(1)); 
draw((3.96,-2.58)--(-2.9,3.62), linewidth(1)); 
draw(circle((-0.1642987913646527,-0.2482080175421804), 4.7378364650868345), linewidth(1)); 
draw((3.158353295416347,3.129228507415229)--(3.96,-2.58), linewidth(1)); 
draw((3.158353295416347,3.129228507415229)--(-4.76,-1.4), linewidth(1)); 
draw((4.338361270565606,1.2259516537976352)--(-4.685181098579763,1.1690843073562132), linewidth(1)); 
 /* dots and labels */
dot((-2.9,3.62),dotstyle); 
label("$A$", (-3.32,3.98), NE * labelscalefactor); 
dot((-4.76,-1.4),dotstyle); 
label("$B$", (-5.34,-1.72), NE * labelscalefactor); 
dot((3.96,-2.58),dotstyle); 
label("$C$", (3.92,-3.14), NE * labelscalefactor); 
dot((3.158353295416347,3.129228507415229),dotstyle); 
label("$D$", (3.34,3.36), NE * labelscalefactor); 
dot((-0.21932072726519639,1.197228645633268),linewidth(4pt) + dotstyle); 
label("$M$", (-0.38,1.64), NE * labelscalefactor); 
dot((-4.685181098579763,1.1690843073562132),dotstyle); 
label("$E$", (-5.32,1.48), NE * labelscalefactor); 
dot((4.338361270565606,1.2259516537976352),linewidth(4pt) + dotstyle); 
label("$F$", (4.56,1.44), NE * labelscalefactor); 
dot((-3.806055407388893,1.1746246531762152),linewidth(4pt) + dotstyle); 
label("$X$", (-4.2,1.36), NE * labelscalefactor); 
dot((3.426404028141226,1.22020439996725),linewidth(4pt) + dotstyle); 
label("$Y$", (3.6,1.44), NE * labelscalefactor); 
clip((xmin,ymin)--(xmin,ymax)--(xmax,ymax)--(xmax,ymin)--cycle); 
 /* end of picture */
    \end{asy}
\end{center}
\end{lemma}

\end{document}