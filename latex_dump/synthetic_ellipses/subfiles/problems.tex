\documentclass[subfile]{main.tex}



\subsection{Problems}

\begin{problem}
Let $\Gamma$ be an ellipse inscribed in a triangle $ABC$ with foci $P$ and $Q$. Prove that $P$, $Q$ are isogonal conjugates with respect to triangle $ABC$.
\end{problem}

\begin{problem}
Let $ABC$ be an acute triangle and $D$ a point on it's interior. Let $E_c$ be the ellipse with focus $D,A$ passing through $C$, $E_b$ the ellipse with focus $D,C$ passing through $B$ and $E_a$ the ellipse with focus $D,B$ passing through $A$. $E_a\cap E_b=P_1,P_2$, $E_a\cap E_c=Q_1,Q_2$, $E_b\cap E_c=R_1,R_2$. Prove that $P_1P_2,Q_1Q_2,R_1R_2$ concur.
\end{problem}

\begin{problem}[Sharygin CR 2019 P21]
An ellipse $\Gamma$ and its chord $AB$ are given. Find the locus of orthocenters of triangles $ABC$ inscribed into $\Gamma$.
\end{problem}

\begin{problem}
Let $S_1$ be one of the focus of an ellipse and $A$ and $B$ be its two vertices. If two parabolas with vertex $A$ and $B$ respectively, having the same focus $S_1$ intersect at points $E$ and $F$, then prove:
\begin{enumerate}
    \item The two parabolas are orthogonal to each other
    \item The line EF passes through the other focus of the ellipse
\end{enumerate}
\end{problem}
\begin{center}
Hide diagram which publishing
\begin{asy}
 /* Geogebra to Asymptote conversion, documentation at artofproblemsolving.com/Wiki go to User:Azjps/geogebra */
import graph; size(15cm); 
real labelscalefactor = 0.5; /* changes label-to-point distance */
pen dps = linewidth(0.7) + fontsize(10); defaultpen(dps); /* default pen style */ 
pen dotstyle = black; /* point style */ 
real xmin = -7.064468507799478, xmax = 6.635714970028591, ymin = -2.165316580217529, ymax = 3.8924512240344575;  /* image dimensions */
pen qqffff = rgb(0,1,1); pen qqwuqq = rgb(0,0.39215686274509803,0); 

draw((0.2508048393645657,-1.6586913852987095)--(0.05587252143069951,-1.5734474243149683)--(-0.029371439553041723,-1.7683797422488345)--(0.16556087838082445,-1.8536237032325757)--cycle, linewidth(2) + qqwuqq); 
 /* draw figures */draw(shift((-1.0786537614336138,0.8413205969131048))*rotate(0.06095366358184751)*xscale(1.8325952591034413)*yscale(1.3481332114109819)*unitcircle, linewidth(0.8)); 
real parabola1 (real x) {return x^2/2/1.182496636167356;} 
draw(shift((-2.9112479835080944,0.8393710052541006))*rotate(-89.93904633641816)*graph(parabola1,-7.094979817004136,7.094979817004136), linewidth(0.8) + qqffff); /* parabola construction */
real parabola2 (real x) {return x^2/2/6.147884400246407;} 
draw(shift((0.7539404606408668,0.8432701885721087))*rotate(-269.9390463364182)*graph(parabola2,-24.591537600985628,24.591537600985628), linewidth(0.8) + qqffff); /* parabola construction */
draw((xmin, -939.9887477004554*xmin + 153.77173903414598)--(xmax, -939.9887477004554*xmax + 153.77173903414598), linewidth(0.8) + linetype("2 2") + qqwuqq); /* line */
draw((0.5442812965859634,-0.9875817206962215)--(-0.22252866068910637,-2.741090598761004), linewidth(0.4)); 
draw((1.187676807549264,-2.3005953050192782)--(-0.7189256007418362,-1.4668375011490484), linewidth(0.4)); 
draw((-3.502495967016189,0.838742010508201)--(3.827880921281734,0.8465403771442171), linewidth(0.8) + linetype("4 4")); 
 /* dots and labels */
dot((-2.32,0.84),dotstyle); 
label("$S_1$", (-2.240318981565729,0.4423318123412732), NE * labelscalefactor); 
dot((0.16269247713277218,0.8426411938262094),dotstyle); 
label("$S_2$", (0.2870940758373996,0.39218472786898856), NE * labelscalefactor); 
dot((-2.9112479835080944,0.8393710052541005),linewidth(4pt) + dotstyle); 
label("$A$", (-3.2633195048003283,0.3420376433967039), NE * labelscalefactor); 
dot((0.7539404606408672,0.8432701885721086),linewidth(4pt) + dotstyle); 
label("$B$", (1.0693885936050347,0.32197880960779), NE * labelscalefactor); 
dot((0.15982407588472053,3.5389060908849954),linewidth(4pt) + dotstyle); 
label("$E$", (0.6983001685101308,3.3207744610504126), NE * labelscalefactor); 
dot((0.16556087838082445,-1.8536237032325757),linewidth(4pt) + dotstyle); 
label("$F$", (-0.40493568988012363,-2.0549929943785026), NE * labelscalefactor); 
clip((xmin,ymin)--(xmin,ymax)--(xmax,ymax)--(xmax,ymin)--cycle); 
 /* end of picture */
\end{asy}
    
\end{center}

\begin{definition}
Hello world this
\end{definition}

\begin{problem}
In an ellipse $\varepsilon$, let the intersection of tangents to the extremities of the major axis meet the tangent to a variable point $P$ on the ellipse at $A$ and $B$. Prove cyclicity between $ABF_1F_2$, where $F_1$ and $F_2$ are the two foci.
\end{problem}

\begin{problem}[ISL 1969 Belgium]
Find the equations of regular hyperbolas passing through the points $A(\alpha, 0), B(\beta, 0),$ and $C(0, \gamma).$
$(b)$ Prove that all such hyperbolas pass through the orthocenter $H$ of the triangle $ABC.$
$(c)$ Find the locus of the centers of these hyperbolas.
$(d)$ Check whether this locus coincides with the nine-point circle of the triangle $ABC.$
\end{problem}

\begin{problem}[ISL 1969 Belgium]
Let $O$ be a point on a nondegenerate conic. A right angle with vertex $O$ intersects the conic at points $A$ and $B$. Prove that the line $AB$ passes through a fixed point located on the normal to the conic through the point $O.$
\end{problem}
\begin{problem}[ISL 1969 Belgium]
Let $G$ be the centroid of the triangle $OAB.$
$(a)$ Prove that all conics passing through the points $O,A,B,G$ are hyperbolas.
$(b)$ Find the locus of the centers of these hyperbolas.
\end{problem}

\begin{problem}[Sharygin finals 2017]
If $ABC$ is acute triangle, prove distance from each vertex to corresponding excentre is less than sum of two greatest side of triangle
\end{problem}
\begin{remark}
In this problem, we have a simple triangle inequality popping, but the author (wizardmath) also talked about something called "animation with point $A$ to locate an ellipse" (thanks anant mudgal)
\end{remark}
\begin{problem}[2014 KoMaL Hungary]
The konvex quadrilateral $ABCD$ has an inscribed circle with center $I$. The rays $AB$ and $DC$ meet at point $F$, the rays $AD$ and $BC$ meet at point $G$. Let E be the ellipse with foci $F$ and $G$ that passes through points $B$ and $D$, and let H be the hyperbola branch with foci $F$ and $G$ that passes through points $A$ and $C$. Denote by $P$ and $Q$ the intersections of E and H. Show that the points $P, Q$ and $I$ are collinear.
\end{problem}
\begin{problem}[IGMO 2020 Round 2 P5]
A non-rectangular trapezoid is called a Pepe Trapezoid if
$(a)$ it has integral side lengths, and
$(b)$ An ellipse that is not a circle with integral lengths of semi-major axis and semi-minor axis can be inscribed in the trapezoid such that the major axis or minor axis of the ellipse is perpendicular to the bases of the trapezoid.

Prove or disprove that there are infinitely many non-similar Pepe trapezoids.
\end{problem}
\begin{problem}[Oral Moscow Team MO 2021 X-XI p6]
In the tangential quadrilateral $ABCD$, the point $I$ is the center of the inscribed circle. Rays $AB$ and $DC$ meet at $F$, and rays $AD$ and $BC$ meet at $G$ . Ellipse $\omega$ with focuses at $F$ and $G$ passes through points$ B$ and $D$. The branch of the hyperbola $\gamma$ with focuses at $F$ and $G$ passes through the points $A$ and $C$. Curves $\omega$ and $\gamma$ meet at points $P$ and $Q$. Prove that points $I,P$ and $Q$ are collinear
\end{problem}
\begin{problem}[2020 Instagram Math Olympiad IGMO - Shortlist g3]
For any given ellipse $\omega$ , let P be a point external to it. $A$ and $B$ are points on $\omega$ such that $P A$ and $P B$ are tangent to $\omega$ . C, D $\omega$ and E are points on $\omega$ such that $AC = CD = DE = EB$. Lines $P C, P D, P E$ meets $\omega$ again at points $F, G, H$, respectively. $M_1, M_2, M_3$ are midpoints of $CF, DG$ and $EH$, respectively. Prove that $P, A, B, M_1, M_2$ and $M_3$ lie on an ellipse.
\end{problem}
