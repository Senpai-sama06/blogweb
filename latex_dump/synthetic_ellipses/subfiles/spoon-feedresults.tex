\documentclass[subfile]{main.tex}


\subsection{Spoon-fed Results}

\begin{ques}
What could be the shortest path taken to cover the distance starting from $A$ touching the line $l$ and then making it to $B$? 
\end{ques}
\begin{center}
\begin{asy}
 /* Geogebra to Asymptote conversion, documentation at artofproblemsolving.com/Wiki go to User:Azjps/geogebra */
import graph; size(7cm); 
real labelscalefactor = 0.5; /* changes label-to-point distance */
pen dps = linewidth(0.7) + fontsize(10); defaultpen(dps); /* default pen style */ 
pen dotstyle = black; /* point style */ 
real xmin = -10.332463778655223, xmax = 7.2033396610603555, ymin = 0.5849814756685627, ymax = 11.414864172589047;  /* image dimensions */

 /* draw figures */
draw((xmin, 0.4153606776815223*xmin + 6.622038466424935)--(xmax, 0.4153606776815223*xmax + 6.622038466424935), linewidth(2) + dotted); /* line */
draw((-7.169121460800589,2.0588208837460398)--(-4.82,4.62), linewidth(0.8)); 
draw((-4.82,4.62)--(1.5583344829415946,4.357050948931486), linewidth(0.8)); 
draw((-4.82,4.62)--(-0.5049701623637175,9.324552101959146), linewidth(0.8) + linetype("4 4")); 
draw((1.5583344829415946,4.357050948931486)--(-3.2902527682277656,5.255396846870348), linewidth(0.8)); 
draw((-3.2902527682277656,5.255396846870348)--(-7.169121460800589,2.0588208837460398), linewidth(0.8)); 
 /* dots and labels */
dot((-7.169121460800589,2.0588208837460398),dotstyle); 
label("$A$", (-6.876016875545781,1.517355482688207), NE * labelscalefactor); 
dot((1.5583344829415946,4.357050948931486),dotstyle); 
label("$B$", (2.0752826224523235,4.099314271357992), NE * labelscalefactor); 
dot((-4.82,4.62),dotstyle); 
label("$C$", (-5.3657791643390516,4.924106662183061), NE * labelscalefactor); 
dot((-0.5049701623637175,9.324552101959146),dotstyle); 
label("$B'$", (-0.3632340112913559,9.693558313475858), NE * labelscalefactor); 
dot((-3.2902527682277656,5.255396846870348),dotstyle); 
label("$D$", (-3.178286301716045,5.605456898082032), NE * labelscalefactor); 
clip((xmin,ymin)--(xmin,ymax)--(xmax,ymax)--(xmax,ymin)--cycle); 
 /* end of picture */
\end{asy}
\end{center}

Upon reflecting $B$ over the line $l$, we obtain $B'$. Now the shortest distance would be achieved when the path $ACB'$ becomes a straight line.

Using a similar approach we directly induce our next theorem

\begin{theorem}[Optical property of an ellipse]
Suppose a line $\ell$ is tangent at point $P$ to an ellipse. Then line $\ell$ is the exterior angle bisector of $\angle F_1PF_2$
\end{theorem}
\begin{center}
    \begin{asy}
     /* Geogebra to Asymptote conversion, documentation at artofproblemsolving.com/Wiki go to User:Azjps/geogebra */
import graph; size(7cm); 
real labelscalefactor = 0.5; /* changes label-to-point distance */
pen dps = linewidth(0.7) + fontsize(10); defaultpen(dps); /* default pen style */ 
pen dotstyle = black; /* point style */ 
real xmin = -10.82105927256238, xmax = 6.374268055136586, ymin = -4.62160475634343, ymax = 10.555251755771268;  /* image dimensions */

 /* draw figures */draw(shift((-1.81459139289836,3.3628525837842513))*rotate(0.3274008908443914)*xscale(3.4358794798226366)*yscale(2.6090194267187443)*unitcircle, linewidth(0.8)); 
draw((xmin, 1.3729964016421083*xmin + 11.23784265591496)--(xmax, 1.3729964016421083*xmax + 11.23784265591496), linewidth(2) + dotted); /* line */
draw((-4.05023944813411,3.350077452040048)--(-4.82,4.62), linewidth(0.8)); 
draw((-4.82,4.62)--(0.42105666233738925,3.3756277155284566), linewidth(0.8)); 
draw((-4.82,4.62)--(-7.612257948768518,9.226564106547425), linewidth(0.8) + linetype("4 4")); 
 /* dots and labels */
dot((-4.05023944813411,3.350077452040048),dotstyle); 
label("$A$", (-3.9480383941804758,3.605580086924134), NE * labelscalefactor); 
dot((0.42105666233738925,3.3756277155284566),dotstyle); 
label("$B$", (0.5232577162910236,3.6311303504125423), NE * labelscalefactor); 
dot((-4.82,4.62),dotstyle); 
label("$C$", (-4.714546298832733,4.883093261344563), NE * labelscalefactor); 
dot((-7.612257948768518,9.226564106547425),dotstyle); 
label("$B'$", (-7.499525019069266,9.482140689258108), NE * labelscalefactor); 
clip((xmin,ymin)--(xmin,ymax)--(xmax,ymax)--(xmax,ymin)--cycle); 
 /* end of picture */
    \end{asy}
\end{center}

 In the previous experiment, we wanted to minimize the total length by reflecting about the line so as to obtain the straight-est path possible. In this case however it is the opposite. We are given a point $C$ on the ellipse and a tangent running through the point. Since this is exactly the same situation as the aforementioned (question) we are highly motivated to reflect a center across the tangent. As it is an ellipse, we already are present with the "straightest line" condition, hence we are done.

\begin{exercise}
Point of contact of parallel tangents to an ellipse pass through the center of the ellipse.
\end{exercise}


\begin{theorem}[isogonal conjugates]
For any ellipse the tangents at points $X_1,X_2$ from a point has the focii $F_1.F_2$ as its isogonal conjugates
\end{theorem}


\begin{theorem}
Suppose $PQ$ is a chord of the ellipse containing $F_1$ and let the tangents from $P$ and $Q$ intersect at $R$. Then the excircle of triangle $F_1PQ$ has the excenter opposite to $F_1$ to be $R$.
\end{theorem}







For the reference of the above theorem, here is a brief about isogonal conjugates.


\begin{lemma}[The director circle]
The locus of perpendicular tangents to an ellipse from a point is a circle. 
\end{lemma}


\begin{theorem}[Common chord intersctions]
Let $A_{i,j}$ and $B_{i,j}$ for $1\leq i\leq j\leq 3$ be the points of intersection of ellipses then $\epsilon_i,\epsilon_j$. Then the lines $A_{1,2}B_{1,2},A_{2,3}B_{2,3},A_{3,1}B_{3,1}$ are concurrent.
\end{theorem}
\begin{center}
    \begin{asy}
     /* Geogebra to Asymptote conversion, documentation at artofproblemsolving.com/Wiki go to User:Azjps/geogebra */
import graph; 
size(7cm); 
real labelscalefactor = 0.5; /* changes label-to-point distance */
pen dps = linewidth(0.7) + fontsize(10); defaultpen(dps); /* default pen style */ 
pen dotstyle = black; /* point style */ 
real xmin = -15.59413416595416, xmax = 27.09118576920159, ymin = -10.28758315092448, ymax = 8.586452896538217;  /* image dimensions */

 /* draw figures */draw(shift((0.9987515959479094,0.33))*rotate(-1.4858001381307109)*xscale(4.5674052124427496)*yscale(3.6841593611050265)*unitcircle, linewidth(0.8)); draw(shift((4.12750319189582,-1.94))*rotate(-78.94070363293243)*xscale(4.523886252790249)*yscale(3.9294588467350766)*unitcircle, linewidth(0.8)); draw(shift((5.0287515959479085,1.86))*rotate(50.23851940220572)*xscale(3.6896883716273314)*yscale(3.0465682277655803)*unitcircle, linewidth(0.8)); 
draw((1.7515065217038295,1.7978510790635842)--(7.53850445647632,0.12612416436045404), linewidth(0.8)); 
draw((2.641418802211332,3.7531070784150136)--(4.964827037121927,-1.5326898908742206), linewidth(0.8)); 
draw((0.43486712150317164,-3.32128985183861)--(4.677503191895819,2.48), linewidth(0.8)); 
 /* dots and labels */
dot((-1.7,0.4),dotstyle); 
dot((3.6975031918958186,0.26),dotstyle); 
label("$O$", (3.248653477456468,-0.9130619352973116), NE * labelscalefactor); 
dot((4.557503191895819,-4.14),dotstyle); 
dot((6.36,3.46),dotstyle); 
dot((0.43486712150317164,-3.32128985183861),linewidth(4pt) + dotstyle); 
label("$F$", (-0.3761613925860405,-4.319137976975182), NE * labelscalefactor); 
dot((2.641418802211332,3.7531070784150136),linewidth(4pt) + dotstyle); 
label("$A$", (2.029965719424935,4.555408773818536), NE * labelscalefactor); 
dot((4.964827037121927,-1.5326898908742206),linewidth(4pt) + dotstyle); 
label("$B$", (5.092309316529813,-1.2880427839223982), NE * labelscalefactor); 
dot((1.7515065217038295,1.7978510790635842),linewidth(4pt) + dotstyle); 
label("$D$", (0.8737747694975833,2.1180332577554726), NE * labelscalefactor); 
dot((7.53850445647632,0.12612416436045404),linewidth(4pt) + dotstyle); 
label("$C$", (8.467136954155597,-0.25684545020340993), NE * labelscalefactor); 
dot((4.677503191895819,2.48),linewidth(4pt) + dotstyle); 
label("$E$", (5.061060912477723,2.992988571214008), NE * labelscalefactor); 
clip((xmin,ymin)--(xmin,ymax)--(xmax,ymax)--(xmax,ymin)--cycle); 
 /* end of picture */
    \end{asy}

    
\end{center}


{\sffamily Following from the above theorem, with the same configurations we have another remarkable result, dealing with the common tangents of the three ellipses}
\begin{theorem}
Given three confocal ellipses, the intersections of the common tangents of two of the ellipses would be collinear.
\end{theorem}
Here is the diagram to the 

\centering
\includegraphics[scale=0.3]{ilyabogdanov.jpg}


\begin{lemma}[Auxillary circle]
In an ellipse, with foci $F_1 \text{ and }F_2$, constract a tangent line from point $P$ on the ellipse. If the tangent meets the auxillary circle at points $C$ and $D$ then :-
\begin{itemize}
    \item $OC\| PF_1$ and $OD\| PF_2$
    \item $\overline{F_1D}\cdot \overline{F_2C}=\overline{AB}$
\end{itemize}
\begin{center}
    \begin{asy}
    import graph; size(10cm);
real labelscalefactor = 0.5; /* changes label-to-point distance */
pen dps = linewidth(0.7) + fontsize(10); defaultpen(dps); /* default pen style */
pen dotstyle = black; /* point style */
real xmin = -4.030574175810688, xmax = 10.53715615317201, ymin = -2.04073603838753, ymax = 4.40063227691669; /* image dimensions */
pen ccqqqq = rgb(0.8,0,0);
/* draw figures */draw(shift((2.32,0.78))*rotate(-0.5968094512291775)*xscale(2.2241103603546306)*yscale(1.1224379248033305)*unitcircle, linewidth(0.8));
draw((0.4,0.8)--(0.6546701778119229,2.2542263999697707), linewidth(0.8) + ccqqqq);
draw((4.24,0.76)--(4.387204035363257,1.6005695407548783), linewidth(0.8) + ccqqqq);
draw((0.4,0.8)--(3.02,1.84), linewidth(0.8) + ccqqqq);
draw((4.24,0.76)--(3.02,1.84), linewidth(0.8) + ccqqqq);
draw((0.09601029553611135,0.8031665594214987)--(4.543989704463887,0.7568334405785011), linewidth(0.8));
draw(circle((2.32,0.78), 2.2241103603546293), linewidth(0.8) + linetype("4 4"));
draw((5.963129097047367,1.324586986947784)--(-0.568672586312144,2.468463274827004), linewidth(0.8));
/* dots and labels */
dot((0.4,0.8),dotstyle);
label("$F_1$", (0.4058649287696944,0.4440964408606208), NE * labelscalefactor);
dot((4.24,0.76),dotstyle);
label("$F_2$", (3.9251555646147094,0.41210288962566605), NE * labelscalefactor);
dot((3.02,1.84),dotstyle);
label("$P$", (2.976013544644387,2.001115934295085), NE * labelscalefactor);
dot((0.6546701778119229,2.2542263999697707),linewidth(4pt) + dotstyle);
label("$D$", (0.3845358946130579,2.4810192028194056), NE * labelscalefactor);
dot((4.387204035363257,1.6005695407548783),linewidth(4pt) + dotstyle);
label("$C$", (4.426387867295666,1.6811804219455373), NE * labelscalefactor);
dot((4.543989704463887,0.7568334405785011),linewidth(4pt) + dotstyle);
label("$B$", (4.895626618741668,0.6253932311920309), NE * labelscalefactor);
dot((0.09601029553611135,0.8031665594214987),linewidth(4pt) + dotstyle);
label("$A$", (-0.27666416424267215,0.5507416116438032), NE * labelscalefactor);
dot((2.32,0.78),linewidth(4pt) + dotstyle);
label("$O$", (2.35747155410193,0.8600126069150323), NE * labelscalefactor);
clip((xmin,ymin)--(xmin,ymax)--(xmax,ymax)--(xmax,ymin)--cycle);
/* end of picture */
    \end{asy}
\end{center}
\end{lemma}

\begin{corollary}
Following from the above lemma, the circle having $PF_1$ or $PF_2$ as diameter has one and only one common ponit with the auxillary circle, and that is $C$ or $D$ respectively.
\end{corollary}
