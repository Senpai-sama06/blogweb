\documentclass[11pt, a4paper]{article}

% Packages for formatting and math
\usepackage[utf8]{inputenc}
\usepackage{geometry}
\usepackage{amsmath, amssymb}
\usepackage{graphicx}
\usepackage{hyperref}
\usepackage{xcolor}
\usepackage{titlesec}
\usepackage{fancyhdr}

% Page geometry settings
\geometry{top=1in, bottom=1in, left=1in, right=1in}

% Custom colors for the blog post feel
\definecolor{darkblue}{RGB}{25, 50, 120}
\definecolor{accent}{RGB}{200, 50, 50}

% Title formatting
\titleformat{\section}{\Large\bfseries\color{darkblue}}{}{0em}{}
\titleformat{\subsection}{\large\bfseries\color{darkblue}}{}{0em}{}

% Header setup
\pagestyle{fancy}
\fancyhf{}
\rhead{\textit{Deep Dive: Speech Enhancement}}
\lhead{Blog Post Draft}
\cfoot{\thepage}

\title{\textbf{\Huge Saving the Voice: \\ \Large A Deep Dive into Harmonic Regeneration for Speech Enhancement}}
\author{\textit{Based on "Speech Enhancement Using Harmonic Regeneration" (Plapous et al., 2005)}}
\date{}

\begin{document}

\maketitle

\noindent
\textbf{TL;DR:} Traditional noise reduction often kills the "color" of human speech by accidentally suppressing weak harmonics. This post explores a clever 2005 technique called \textit{Harmonic Regeneration Noise Reduction} (HRNR) that uses non-linear distortion to "hallucinate" missing harmonics back into existence, creating a roadmap for cleaner, more natural audio.

\hrulefill

\section{The Problem: The "Is it Noise or Speech?" Dilemma}

If you have ever used a basic noise reduction plugin, you know the sound: the background hiss disappears, but the voice suddenly sounds "muffled," "robotic," or "underwater."

Why does this happen? The culprit is usually the \textbf{Signal-to-Noise Ratio (SNR)} estimation.

Classical algorithms (like Wiener filtering or Spectral Subtraction) work on a simple logic for every frequency bin $\omega_k$:
\begin{itemize}
    \item \textbf{High SNR:} Keep the signal.
    \item \textbf{Low SNR:} Suppress the signal.
\end{itemize}

The problem arises with human speech. Vowels are rich in \textit{harmonics}—energy peaks at integer multiples of a fundamental frequency (pitch). However, the higher harmonics often have very low energy. 

In a noisy room, these weak high-frequency harmonics get buried. The algorithm looks at them, sees a low SNR, assumes they are random noise, and deletes them. The result? You lose the bright, human quality of the voice.

\section{The Intuition: How to Resurrect a Harmonic}

The paper \textit{"Speech Enhancement Using Harmonic Regeneration"} proposes a fascinating solution. Instead of trying to fish the weak harmonics out of the noise (which is impossible if they are buried deep), we simply \textbf{regenerate} them artificially.

The core intuition relies on a property of periodic signals: \textbf{Periodicity in Time $\Leftrightarrow$ Harmonics in Frequency.}

If a signal repeats every $T$ seconds, its spectrum \textit{must} have peaks at frequencies $f, 2f, 3f \dots$ (where $f = 1/T$). Even if the shape of the wave is damaged, as long as the \textit{periodicity} remains, the potential for harmonics exists.

The authors propose a "two-pass" system:
\begin{enumerate}
    \item \textbf{Denoise roughly:} Do a standard noise reduction. This removes noise but damages the speech harmonics.
    \item \textbf{Distort intentionally:} Apply a non-linear function to this damaged signal.
\end{enumerate}

\subsection*{The Magic of Non-Linearity}
Why distortion? Imagine a sine wave (a single tone). If you chop off the bottom half (half-wave rectification), it is no longer a sine wave. It becomes a complex periodic shape.

Mathematically, applying a non-linearity in the time domain is equivalent to \textbf{convolving} the spectrum in the frequency domain. This operation spreads energy from the strong low frequencies (which survived the first pass) up into the higher frequencies.

\begin{quote}
    \textit{Key Insight: By intentionally distorting the dominant low-frequency voice, we artificially create energy at the exact high-frequencies where the original voice used to be.}
\end{quote}

\section{The Math: From Time to Frequency}

Let's look at the "Harmonic Regeneration" step formally.
Let $\hat{s}(t)$ be our roughly denoised speech signal (where harmonics are missing). We create an artificial harmonic signal $s_{harmo}(t)$ using a non-linear function $NL(\cdot)$:

$$ s_{harmo}(t) = NL(\hat{s}(t)) $$

The paper suggests a simple function: the \textbf{Maximum relative to zero} (essentially keeping only positive values):

$$ s_{harmo}(t) = \max(\hat{s}(t), 0) $$

\subsection*{Why does this work?}
We can model the "rectified" signal as the original signal multiplied by a pulse train $p(t)$ that is 1 when the signal is positive and 0 otherwise. Since the voice is periodic with period $T$, $p(t)$ is also periodic with period $T$.

In the frequency domain, the Fourier Transform (FT) of a periodic pulse train is a "comb" of impulses spaced by the fundamental frequency $1/T$.

$$ \text{Spectrum}(s_{harmo}) = \text{Spectrum}(\hat{s}) * \text{Comb Function} $$

This convolution copies the spectrum of the speech and shifts it by the fundamental frequency multiples. \textbf{Result:} The gaps in the high frequencies are filled in with new energy that is perfectly aligned with the speaker's pitch.

\section{The Solution: The HRNR Algorithm}

We don't actually output the distorted artificial signal $s_{harmo}(t)$ to the listener (that would sound buzzy and harsh). We only use it as a \textbf{guide}.

The algorithm calculates a new \textit{Harmonic SNR} ($SNR_{harmo}$) based on this artificial signal. It essentially asks: \textit{"Does the artificial signal have strong energy here?"}

If the answer is \textbf{yes}, the filter concludes: \textit{"This frequency bin corresponds to a harmonic. Even if the original input looks noisy, I should preserve it."}

The final gain function $G_{harmo}$ becomes a mix of the classical estimate and this new harmonic map:

$$ SNR_{harmo}(p, \omega_k) = \frac{\rho |\hat{S}(p, \omega_k)|^2 + (1-\rho)|S_{harmo}(p, \omega_k)|^2}{\text{Noise Power}} $$

Where $\rho$ is a mixing factor (typically 0.5).

\section{Conclusion}

The Harmonic Regeneration Noise Reduction (HRNR) technique is a brilliant example of using \textbf{prior knowledge} about a signal (that human speech is periodic) to solve a blind estimation problem.

By temporarily destroying the signal (via non-linear distortion) to recover its structure, and then using that structure to guide the final filter, we get the best of both worlds: effective noise suppression and the preservation of the rich, natural harmonics that make us sound human.

\end{document}